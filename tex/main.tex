\documentclass[12pt, a4paper]{article}

%% Text related
\usepackage[utf8]{inputenc}
\usepackage{indentfirst}
\usepackage[bottom]{footmisc}
\usepackage{verbatim} %Adds comment blocks & code copy-paste
\usepackage[ngerman, num]{isodate} %Proper date format ffs...
    \monthyearsepgerman{}{}
    \daymonthsepgerman{}{}
\usepackage{enumitem}
\usepackage{lipsum} %Because why not?
\usepackage{lscape}

%% Figure related
\usepackage{graphicx}
\usepackage{float} 
\usepackage{subfig}
\usepackage[justification=centering]{caption} %Centers multiline captions

%% Drawing stuff
\usepackage{pgfplots}
\usepackage[american]{circuitikz}

%% Math & Scientific notations
\usepackage{amsmath}
\usepackage{amssymb} %amsmath doesn't have quite common math symbols for some reason
\usepackage[per-mode=repeated-symbol, tight-spacing=true]{siunitx}

%% Table related
\usepackage{multirow}
\usepackage{tabularx}
\usepackage[thinlines]{easytable} %convenient
\usepackage{booktabs} %for different horizontal lines
\usepackage{array} %for table corners not meeting up - doesn't appear to work though

%%
\usepackage{subfiles}

% pgfplots configurations
\graphicspath{ {../img/} }
\pgfplotsset{
    compat=newest,
    standard/.style={
    axis x line=middle,
    axis y line=middle,
    every axis x label/.style={at={(current axis.right of origin)},anchor=west},
    every axis y label/.style={at={(current axis.above origin)},anchor=south}
    }
}

% To place e.g. [ V ] so I don't bother typing it
\newcommand{\unitV}{{\;[\,\SI{}{\volt}\,]}}
\newcommand{\unitmA}{{\;[\,\SI{}{\milli\ampere}\,]}}
\newcommand{\unitA}{{\;[\,\SI{}{\ampere}\,]}}
\newcommand{\unitohm}{{\;[\,\SI{}{\ohm}\,]}}
\newcommand{\unitkohm}{{\;[\,\SI{}{\kilo\ohm}\,]}}

% \title{EED 3009 ENGINEERING DESIGN - II\\FEASIBILITY REPORT}
% \author{Abdurrahman ÜZÜM}
% \date{\today}




\begin{document}

    \begin{titlepage}
        \begin{center}

            \begin{figure}

                \subfloat
                {%
                    \includegraphics[width=0.3\textwidth]{deulogo.png}
                }
                %
                \hfill
                %
                \subfloat
                {%
                    \includegraphics[width=0.3\textwidth]{facultylogo.png}
                }

            \end{figure}

            \textbf{T.C.\\}
            \textbf{DOKUZ EYLUL UNIVERSITY\\}
            \textbf{ENGINEERING FACULTY\\}

            \vspace*{1 cm}
            \textbf{ELECTRICAL \& ELECTRONICS ENGINEERING\\}
            \textbf{DEPARTMENT\\}

            \vspace*{1 cm}
            \textbf{EED3009 ENGINEERING DESIGN - II\\}
            \textbf{THEORETICAL DESIGN REPORT\\}

            \textbf{RANGE FINDER}

            \vspace*{1 cm}

        
            \begin{table}[H]\centering
                \begin{tabular}{cc}
                    Elif Sezin ÖZYİĞİT \hspace{1cm}  & \hspace{1cm} Abdurrahman ÜZÜM \\
                    2019502098         \hspace{1cm}  & \hspace{1cm} 2019502099       \\             
                \end{tabular}
            \end{table}

            Instructor:\\ Dr.Ogr.Uy.Neslihan AVCU\\

            \vspace*{1 cm}
            December, 2022


        \end{center}

    \end{titlepage}

    \pagebreak
    
    \tableofcontents

    \pagebreak

    \begin{abstract}
        Early societies measured distance with a variety of primitive tools, from basic paces to measuring rods and marked ropes.\cite{abstract} Luckily, we’ve come a long way from the days of using belts, thumbs and cubits for measurement. Various methods have been developed over the years in order to increase the measurement accuracy and to be able to measure in various conditions. These devices, which have been developed by human beings step by step over the years and evolved with new technologies, have reached the level where they can measure without the need for physical contact or even light.
    \end{abstract}

    \section{Introduction}

        This project focuses on measuring distances with no contact up to a meter. To realise this project a battery powered, handheld device will be designed. The device will utilise an ultrasonic speaker and microphone. It will send regular periodic ultrasonic bursts from the speaker and listen for the echoes. The necessary circuitry will measure the time between sent ultrasonic bursts and their respective echoes, following the time of flight (ToF) principle. This data will then be processed using necessary information such as the speed of sound on air, as a result of which, the distance data will be obtained. Finally, this distance value will be printed on a two digit display. The device will also employ a laser guide to assist proper alignment of the sensors, and also to provide a feedback to the user. 


        \section{Objectives of the Project}

            \noindent The designed device must follow the above guidelines and accomplish these aims. 
            \begin{itemize}
                \item Measure distances in the range of 0-99\SI{}{\centi\metre}
                \item Have an accuracy of \SI{1}{\centi\metre}
                \item Be battery powered and handheld
                \item Display the measured distance in real time on a two digit display.
                \item Not empoly an MCU or ASIC designed for this particular purpuse.
            \end{itemize}

            \noindent In terms of measurement accuracy and range, the given properties can evolve as the experiments and tests continue, and therefore are subject to change. 

        
    \pagebreak
    \section{Overview of the Project}

       
        \pagebreak
        
          \section{Methodology}
        
        Guidelines of distance measurement using ultrasonic sensors is relatively straight forward. All the device should do is to send sound signals, wait for the echoes arrive and measure the time between. Since transmitting a continuous signal would produce a continuous echo, it is not possible to keep track of the time. Therefore the transmitter should emit bursts of sound waves, and the receiver should keep track of the time between.

        \bigskip 

        Two ultrasonic transducers are needed, one for the transmitter and one for the receiver. The transmitter should be driven periodically with short pulses. The timing between these pulses is critical, such that the following pulse should not be transmitted before the echo of the previous pulse has arrived. This calculation should be made in consideration of the longest distance to be measured. Width of these pulses is also critical, as it should be short enough to ensure that the echo doesn't arrive before the pulse ends. This calculation should be made in considetation of the shortest distance of interest. 

        Assuming that the longest distance to be measured is \SI{100}{\centi\metre}, and speed of sound is \SI{343}{\metre\per\second}, the echo would arrive after:

        \begin{equation}
            \texttt{max }t_f = 2 \times \frac{\SI{1}{\metre}}{\SI{343}{\metre\per\second}} = \SI{5.83}{\milli\second}
        \end{equation}

        \noindent Therefore the time between end of a pulse and start of the other should be around \SI{6}{\milli\second} minimum. 

        \noindent Assuming that the shortest distance to be measured is \SI{1}{\centi\metre}, 

        \begin{equation}
            \texttt{min }t_f = 2 \times \frac{\SI{0.01}{\metre}}{\SI{343}{\metre\per\second}} = \SI{58.3}{\micro\second}
        \end{equation}

        \noindent Therefore the width of the pulse should not exceed about \SI{50}{\micro\second}.


        \bigskip
        Commonly available cheap ultrasonic transducers work at frequencies near \SI{40}{\kilo\hertz}. Therefore the transmitter should send bursts of \SI{40}{\kilo\hertz} sound waves, filling in the previously described pulse. In order to provide a signal with these properties, two generators are needed. One to generate the \SI{40}{\kilo\hertz} signal, and another to generate the pulses. The \SI{40}{\kilo\hertz} signal should then be connected to the transmitter via an analogue switch, which is driven by the pulses.

        \pagebreak
        On the receiver side, the received signal will most likely be very low in amplitude, therefore the first stage is to amplify it. This amplification should be made such that the lowest possible echo should not fall short of the limits of the next stage, and the highest possible amplitude echo should not exceed it.
        
        \bigskip
        After the amplification, the \SI{40}{\kilo\hertz} sound burst should be converted into a regular pulse with digital voltage levels. To accomplish this, firstly the signal should be passed though a lowpass filter with appropriate passband frequency. After which, the signal will resemble a single pulse, albeit with slow-rising edges. Feeding this signal into a logical buffer/schmitt-trigger, the pulse can be converted into a true digital pulse. 

        \bigskip
        After converting the echo burst into a digital pulse, timing measurements can be done. A counter that is to be started simultaneously with the first edge of the transmitted burst, should then be stopped by the first edge of the echo signal. Output of the counter is proportional to the distance, however, it is necessary to convert it into proper units. For which an arithmetic circuit should follow the counter.

        \bigskip
        After obtaining the meaningful distance data, it then can be transferred to the 7-segment displays. A binary to 7 segment display driver can be employed here for simplicity.

        \pagebreak
        \begin{landscape}\centering
            \vspace*{1.8cm}
            \begin{figure}[H]\centering
                \makebox[\linewidth]
                {
                    \includegraphics[width=1.2\linewidth]{block2.png}
                }
                \caption{Block Diagram}
            \end{figure}
        \end{landscape}
        \vfill
        \pagebreak


    \section{Technical Feasibility}

        Since HCSR-04 is a cheap and commonly available product it is easier to desolder and salvage the ultrasonic transduces out of it, instead of looking for parts. However, the transducers are not labelled, marked or named, therefore it is not possible to find relevant information about them. Instead, a working HCSR-04 module is put under test and the signal driving the transmitter of the device is observed with an oscilloscope. 

        \noindent On these images, each channel of the scope is connected to one pin of the transmitter, and the purple waveform is the difference between them.

        \bigskip

        \begin{figure}[H]\centering
            \includegraphics[width=0.8\textwidth]{experiment1/pulses.png}
            \caption[]{Pulses}\label{fig:pulses}
        \end{figure}

        \noindent As it can be seen, the transmitter is driven with periodic, short pulses. Here, it should be kept in mind that this device uses a microcontroller and tries to be smart, it lenghtens the time between pulses when the distance is longer, and shortens it for shorter distances. Therefore the duration between pulses is not really reliable for this experiement. 

        \begin{figure}[H]\centering
            \includegraphics[width=0.8\textwidth]{experiment1/pulses_close.png}
            \caption[]{Pulses}\label{fig:pulses}
        \end{figure}

        \noindent Here, when zoomed into the short bursts, the \SI{40}{\kilo\hertz} ultrasonic frequency signal can be seen. Note that the transmitter is driven symetrically, but the sensor is powered with positive \SI{5}{\volt}, therefore the module has decided to drive the transmitter with alternating \SI{5}{\volt} square pulses on either pin and obtaining a symmetrical \SI{10}{\volt} square wave. 

        \begin{figure}[H]\centering
            \includegraphics[width=0.8\textwidth]{experiment1/pulses_duration.png}
            \caption[]{Pulses}\label{fig:pulses}
        \end{figure}

        \begin{figure}[H]\centering
            \includegraphics[width=0.8\textwidth]{experiment1/ultrasonic_freq_amplt.png}
            \caption[]{Pulses}\label{fig:pulses}
        \end{figure}



        \pagebreak
        In order to test the behaviours of transducers, they are desoldered from the module and soldered back on an empty prototyping board. It is necessary to measure what the received signal amplitude will be in order to design a proper amplifier.

        $\SI{10}{\volt}_{\texttt{pp}}$ \SI{40}{\kilo\hertz} sinusoidal signal is applied to the transmitter directly from a function generator and output of the receiver is measured using the oscilloscope. Amplitude of the received signal is observed at various distances. 

        It is observed that the received signal amplitude drops below \SI{100}{\milli\volt} for very large distances (not shown on the pictures). At about 40-50 \SI{}{\centi\metre}, the signal amplitude is about \SI{200}{\milli\volt}, and for very close distances at about 5-10 \SI{}{\centi\metre}, amplitude reaches \SI{800}{\milli\volt}. 

        \begin{figure}[H]\centering
            \includegraphics[width=\textwidth]{experiment2/setup.jpg}
            \caption[]{Experiment setup}\label{fig:setup}
        \end{figure}

        \begin{figure}[H]\centering
            \includegraphics[width=0.5\textwidth]{experiment2/near.jpg}
            \caption[]{}\label{fig:near}
        \end{figure}

        \begin{figure}[H]\centering
            \includegraphics[width=\textwidth]{experiment2/near_scope.png}
            \caption[]{}\label{fig:ner_scope}
        \end{figure}


        \begin{figure}[H]\centering
            \includegraphics[width=0.5\textwidth]{experiment2/medium.jpg}
            \caption[]{}\label{fig:medium}
        \end{figure}

        \begin{figure}[H]\centering
            \includegraphics[width=\textwidth]{experiment2/medium_scope.png}
            \caption[]{}\label{fig:medium_scope}
        \end{figure}


        \begin{figure}[H]\centering
            \includegraphics[width=0.5\textwidth]{experiment2/far.jpg}
            \caption[]{}\label{fig:far}
        \end{figure}

        \begin{figure}[H]\centering
            \includegraphics[width=\textwidth]{experiment2/far_scope.png}
            \caption[]{}\label{fig:far_scope}
        \end{figure}


        





    
    \section{Cost Analysis}

        The following table shows the components to be used based on first draft design of the project. All of these componets and given prices are subject to change as the design is going through revisions and further considerations.
        
        \begin{table}[H]\centering
            \includegraphics[width=\textwidth]{cost.png}
            \caption[]{Cost Analysis Table}\label{tab:cost}
        \end{table}

    \section{Risk Feasibility}

        \begin{itemize}
            \item Since there is not a commonly available specific IC to perform necessary calculation and using an MCU is not allowed, this stage can be tricky to realise.
            \item Necessary circuitry may be more complex than expected, such that the required design revisions can take too long.
            \item The received signal may be more noisy than expected, especially in long-range measurements. 
            \item If the adequate precautions and filtrations are not taken, it can be observed that the margin of error may increase more than allowed.
        \end{itemize}   

    \pagebreak
    \section{Gannt Chart}

        \begin{figure}[H]\centering
            \makebox[\linewidth]
                {
                    \includegraphics[width=1.2\linewidth]{ganntchart.png}
                }
                \caption{Gannt Chart}
        \end{figure}



    


    \begin{thebibliography}{}	
        \bibitem{abstract}
        https://www.encyclopedia.com/education/news-wires-white-papers-and-books/distance-measurement

        \bibitem{triangulation}
        https://www.seeedstudio.com/blog/2019/12/23/distance-sensors-types-and-selection-guide/

        \bibitem{tof}
        https://www.sensorpartners.com/en/knowledge-base/everything-about-the-operation-principles-of-ultrasonic-sensors/ 
        
        \bibitem{}
        \textit{Mubina Toa, Akeem Whitehead}, \textbf{\textit{``Ultrasonic Sensing Basics''}}, Texas Instruments
    \end{thebibliography}
            





\end{document}